%\begin{figure}[h!]
\centering
\pgfdeclarelayer{background}
\pgfdeclarelayer{foreground}
\pgfsetlayers{background,main,foreground}
\begin{tikzpicture}[bend angle=30,scale=5,
%    xscale						= 0.4,
%	yscale						= 0.4,
    grow                        = west,
    punkt/.style={rectangle, rounded corners, shade, top color=white, bottom color=blue!50!black!20, draw=blue!40!black!60},
    level 1/.style={sibling distance=1cm,level distance=3cm},
    level 2/.style={sibling distance=4cm, level distance=4cm},
    descr/.style={rounded corners, shade, top color=white, bottom color=green!50!black!20, draw=green!40!black!60, thick },
    lan/.style={rectangle, rounded corners, shade, top color=white, bottom color=green!50!black!20, draw=green!40!black!60, thick},
    wan/.style={rectangle, rounded corners, shade, top color=white, bottom color=red!50!black!20, draw=red!40!black!60, thick},
    lanconn/.style={thick, draw=green!40!black!60, },
    wanconn/.style={thick, draw=red!40!black!60, },
    edge from parent/.style={thick, shorten >=10pt, shorten <=20pt},
	every edge/.style={thick},
    font=\footnotesize,
	%every fit/.style={inner sep=-7pt},
    pre/.style={thick, shorten >=10pt, shorten <=10pt, loosely dotted, <-, draw},
    ]

\node[lan] [grow=left] (lan) {LAN};

\node[lan, node distance=3.5cm] (mail) [right of=lan] {mail.office.axxeo.de}
	edge [lanconn,shorten >=10pt, shorten <=10pt] (lan);

\node[lan,grow=right, node distance=5cm] (firewall) [right of=mail] {firewall.office.axxeo.de}
	edge [lanconn,shorten >=10pt, shorten <=10pt] (mail)
		child [grow=south west] {
			node [wan] (client) {Client}
				edge [lanconn, shorten >=10pt, shorten <=20pt] (firewall)
        }
	    child [grow=north east] {
	    	node[wan] (r1) {ISP-Router eins}
				edge [wanconn,shorten >=10pt, shorten <=20pt] node (alte-verbindung) {} (firewall)
		}
		child [grow=south east] {
			node[wan](r2) {ISP-Router zwei}
				edge [wanconn,shorten >=10pt, shorten <=20pt] node (neue-verbindung) {} (firewall)
		};

\node[wan,wanconn, shape=cloud,cloud ignores aspect,right of=firewall,node distance=4cm] (internet) {Internet}
  edge [wanconn,shorten >=10pt, shorten <=10pt] (r1)
  edge [wanconn,shorten >=10pt, shorten <=10pt] (r2);

%\node [right of=r2, node distance=4cm] (beschr-neu) {neue Verbindung}
%  edge [pre, bend right] (neue-verbindung);
\node [right of=r1, node distance=4cm] (beschr-alt) {alte Verbindung}
  edge [pre, bend left] (alte-verbindung);
%
%
\begin{pgfonlayer}{background}
%  \node [rectangle, rounded corners, shade, top color=white, bottom color=blue!30!black!20,fit=(beschr-neu) ,] {};
  \node [rectangle, rounded corners, shade, bottom color=white, top color=blue!30!black!20,fit=(beschr-alt) ,] {};
  %\node [fill=gray!20,fit=(firewall) (r1) (internet), inner sep=-1mm,label={above:alte Verbindung}] {};
  %\node [fill=gray!50,fit=(firewall) (r2), inner sep=-9mm,label={below:neue Verbindung},circle] {};
\end{pgfonlayer}
%
\end{tikzpicture}
%\vspace{cm}
%\begin{itemize}
%    \item[] \begin{scriptsize}LAN-Verbindung / vertrauenswürdiger Host\end{scriptsize}
%        \tikz {\path[very thick, draw=green!40!black!60,draw] (0,0) -- (4,0){};}
%
%    \item[] \begin{scriptsize}WAN-Verbindung / nicht vertrauenswürdiger Host\end{scriptsize}
%        \tikz {\path[very thick, draw=red!40!black!60,draw] (0,0) -- (4,0) {};}
%\end{itemize}

%    lan/.style={rectangle, rounded corners, shade, top color=white, bottom color=green!50!black!20, draw=green!40!black!60, thick},
%    wan/.style={rectangle, rounded corners, shade, top color=white, bottom color=red!50!black!20, draw=red!40!black!60, thick},
%    lanconn/.style={thick, draw=green!40!black!60, },
%    wanconn/.style={thick, draw=red!40!black!60, },
%\caption{Netzplan}
%\label{fig:netzplan}
%
%\end{figure}
